\apisummary{
    Waits for completion of all outstanding \PUT{}, \acp{AMO}, memory store,
    and non-blocking \PUT{} and \GET{} routines to symmetric data
    objects issued by a \ac{PE}.
}

\begin{apidefinition}

\begin{Csynopsis}
void shmem_quiet(void);
\end{Csynopsis}

\begin{Fsynopsis}
CALL SHMEM_QUIET
\end{Fsynopsis}

\begin{apiarguments}
    \apiargument{None.}{}{}
\end{apiarguments}

\apidescription{ 
    The \FUNC{shmem\_quiet} routine ensures completion of \PUT{}, \acp{AMO},
    memory store, and non-blocking \PUT{} and \GET{} routines on
    symmetric data objects issued by the calling \ac{PE}. All \PUT{}, \acp{AMO},
    memory store, and non-blocking \PUT{} and \GET{} routines to
    symmetric data objects are guaranteed to be completed and visible to all
    \acp{PE} when \FUNC{shmem\_quiet} returns. 
}


\apireturnvalues{
    None.
}

\apinotes{ 
    \FUNC{shmem\_quiet} is most useful as a way of ensuring completion of
    several \PUT{}, \acp{AMO}, memory store, and non-blocking \PUT{}
    and \GET{} routines to symmetric data objects initiated by the calling
    \ac{PE}.  For example, you might use \FUNC{shmem\_quiet} to await delivery
    of a block of data before issuing another \PUT{} or non-blocking
    \PUT{} routine, which sets a completion flag on another \ac{PE}.
     \FUNC{shmem\_quiet} is not usually needed if
    \FUNC{shmem\_barrier\_all} or \FUNC{shmem\_barrier} are called.  The barrier
    routines wait for the completion of outstanding writes (\PUT{}, \ac{AMO},
    memory stores, and nonblocking \PUT{} and \GET{} routines) to
    symmetric data objects on all \acp{PE}.
}

\begin{apiexamples}

\apicexample
    {The following example uses  \FUNC{shmem\_quiet}  in a \CorCpp{} program: }	 
    {./example_code/shmem_quiet_example.c}
    {\VAR{Put1} and \VAR{put2} will be completed and visible before \VAR{put3}
    and \VAR{put4}.}
\end{apiexamples}

\end{apidefinition}
